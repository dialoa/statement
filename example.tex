% Options for packages loaded elsewhere
\PassOptionsToPackage{unicode}{hyperref}
\PassOptionsToPackage{hyphens}{url}
\PassOptionsToPackage{dvipsnames,svgnames,x11names}{xcolor}
%
\documentclass[
  letterpaper,
  DIV=11,
  numbers=noendperiod]{scrartcl}

\usepackage{amsmath,amssymb}
\usepackage{lmodern}
\usepackage{iftex}
\ifPDFTeX
  \usepackage[T1]{fontenc}
  \usepackage[utf8]{inputenc}
  \usepackage{textcomp} % provide euro and other symbols
\else % if luatex or xetex
  \usepackage{unicode-math}
  \defaultfontfeatures{Scale=MatchLowercase}
  \defaultfontfeatures[\rmfamily]{Ligatures=TeX,Scale=1}
\fi
% Use upquote if available, for straight quotes in verbatim environments
\IfFileExists{upquote.sty}{\usepackage{upquote}}{}
\IfFileExists{microtype.sty}{% use microtype if available
  \usepackage[]{microtype}
  \UseMicrotypeSet[protrusion]{basicmath} % disable protrusion for tt fonts
}{}
\makeatletter
\@ifundefined{KOMAClassName}{% if non-KOMA class
  \IfFileExists{parskip.sty}{%
    \usepackage{parskip}
  }{% else
    \setlength{\parindent}{0pt}
    \setlength{\parskip}{6pt plus 2pt minus 1pt}}
}{% if KOMA class
  \KOMAoptions{parskip=half}}
\makeatother
\usepackage{xcolor}
\setlength{\emergencystretch}{3em} % prevent overfull lines
\setcounter{secnumdepth}{5}
% Make \paragraph and \subparagraph free-standing
\ifx\paragraph\undefined\else
  \let\oldparagraph\paragraph
  \renewcommand{\paragraph}[1]{\oldparagraph{#1}\mbox{}}
\fi
\ifx\subparagraph\undefined\else
  \let\oldsubparagraph\subparagraph
  \renewcommand{\subparagraph}[1]{\oldsubparagraph{#1}\mbox{}}
\fi


\providecommand{\tightlist}{%
  \setlength{\itemsep}{0pt}\setlength{\parskip}{0pt}}\usepackage{longtable,booktabs,array}
\usepackage{calc} % for calculating minipage widths
% Correct order of tables after \paragraph or \subparagraph
\usepackage{etoolbox}
\makeatletter
\patchcmd\longtable{\par}{\if@noskipsec\mbox{}\fi\par}{}{}
\makeatother
% Allow footnotes in longtable head/foot
\IfFileExists{footnotehyper.sty}{\usepackage{footnotehyper}}{\usepackage{footnote}}
\makesavenoteenv{longtable}
\usepackage{graphicx}
\makeatletter
\def\maxwidth{\ifdim\Gin@nat@width>\linewidth\linewidth\else\Gin@nat@width\fi}
\def\maxheight{\ifdim\Gin@nat@height>\textheight\textheight\else\Gin@nat@height\fi}
\makeatother
% Scale images if necessary, so that they will not overflow the page
% margins by default, and it is still possible to overwrite the defaults
% using explicit options in \includegraphics[width, height, ...]{}
\setkeys{Gin}{width=\maxwidth,height=\maxheight,keepaspectratio}
% Set default figure placement to htbp
\makeatletter
\def\fps@figure{htbp}
\makeatother
\newlength{\cslhangindent}
\setlength{\cslhangindent}{1.5em}
\newlength{\csllabelwidth}
\setlength{\csllabelwidth}{3em}
\newlength{\cslentryspacingunit} % times entry-spacing
\setlength{\cslentryspacingunit}{\parskip}
\newenvironment{CSLReferences}[2] % #1 hanging-ident, #2 entry spacing
 {% don't indent paragraphs
  \setlength{\parindent}{0pt}
  % turn on hanging indent if param 1 is 1
  \ifodd #1
  \let\oldpar\par
  \def\par{\hangindent=\cslhangindent\oldpar}
  \fi
  % set entry spacing
  \setlength{\parskip}{#2\cslentryspacingunit}
 }%
 {}
\usepackage{calc}
\newcommand{\CSLBlock}[1]{#1\hfill\break}
\newcommand{\CSLLeftMargin}[1]{\parbox[t]{\csllabelwidth}{#1}}
\newcommand{\CSLRightInline}[1]{\parbox[t]{\linewidth - \csllabelwidth}{#1}\break}
\newcommand{\CSLIndent}[1]{\hspace{\cslhangindent}#1}

\KOMAoption{captions}{tableheading}
\usepackage{amsthm}
\theoremstyle{plain}
\newtheorem*{sta_ahlfors_s_lemma}{Ahlfors's Lemma}
\theoremstyle{plain}
\newtheorem{theorem}{Theorem}[section]
\theoremstyle{plain}
\newtheorem{lemma}[theorem]{Lemma}
\theoremstyle{plain}
\newtheorem{corollary}[theorem]{Corollary}
\theoremstyle{plain}
\newtheorem*{remark}{Remark}
\newtheoremstyle{exercise}{1em}{1em}{\itshape}{0pt}{\bfseries}{:}{1em}{}
\theoremstyle{exercise}
\newtheorem{exercise}{Exercise}
\newtheoremstyle{note}{3pt}{3pt}{\normalfont}{0pt}{\itshape}{:}{0.5em}{}
\theoremstyle{note}
\newtheorem{note}{Note}
\newtheoremstyle{break}{9pt}{9pt}{\itshape}{0pt}{\bfseries}{.}{\newline}{}
\theoremstyle{break}
\newtheorem{bthm}{B-Theorem}
\newtheoremstyle{citing}{3pt}{3pt}{\itshape}{0pt}{\bfseries}{.}{0.5em}{}
\theoremstyle{citing}
\newtheorem*{varthm}{}
\makeatletter
\makeatother
\makeatletter
\makeatother
\makeatletter
\@ifpackageloaded{caption}{}{\usepackage{caption}}
\AtBeginDocument{%
\ifdefined\contentsname
  \renewcommand*\contentsname{Table of contents}
\else
  \newcommand\contentsname{Table of contents}
\fi
\ifdefined\listfigurename
  \renewcommand*\listfigurename{List of Figures}
\else
  \newcommand\listfigurename{List of Figures}
\fi
\ifdefined\listtablename
  \renewcommand*\listtablename{List of Tables}
\else
  \newcommand\listtablename{List of Tables}
\fi
\ifdefined\figurename
  \renewcommand*\figurename{Figure}
\else
  \newcommand\figurename{Figure}
\fi
\ifdefined\tablename
  \renewcommand*\tablename{Table}
\else
  \newcommand\tablename{Table}
\fi
}
\@ifpackageloaded{float}{}{\usepackage{float}}
\floatstyle{ruled}
\@ifundefined{c@chapter}{\newfloat{codelisting}{h}{lop}}{\newfloat{codelisting}{h}{lop}[chapter]}
\floatname{codelisting}{Listing}
\newcommand*\listoflistings{\listof{codelisting}{List of Listings}}
\makeatother
\makeatletter
\@ifpackageloaded{caption}{}{\usepackage{caption}}
\@ifpackageloaded{subcaption}{}{\usepackage{subcaption}}
\makeatother
\makeatletter
\@ifpackageloaded{tcolorbox}{}{\usepackage[many]{tcolorbox}}
\makeatother
\makeatletter
\@ifundefined{shadecolor}{\definecolor{shadecolor}{rgb}{.97, .97, .97}}
\makeatother
\makeatletter
\makeatother
\ifLuaTeX
  \usepackage{selnolig}  % disable illegal ligatures
\fi
\IfFileExists{bookmark.sty}{\usepackage{bookmark}}{\usepackage{hyperref}}
\IfFileExists{xurl.sty}{\usepackage{xurl}}{} % add URL line breaks if available
\urlstyle{same} % disable monospaced font for URLs
\hypersetup{
  pdftitle={Statement Examples},
  colorlinks=true,
  linkcolor={blue},
  filecolor={Maroon},
  citecolor={Blue},
  urlcolor={Blue},
  pdfcreator={LaTeX via pandoc}}

\title{Statement Examples}
\author{}
\date{}

\begin{document}
\maketitle
\begin{abstract}
This example document recreates in markdown the AMS LaTeX package's
``Newtheorem andtheoremstyle test'' file (\texttt{thmtest.pdf}) by
Michael Downes and Barbara Beeton.
\end{abstract}
\ifdefined\Shaded\renewenvironment{Shaded}{\begin{tcolorbox}[sharp corners, enhanced, interior hidden, frame hidden, borderline west={3pt}{0pt}{shadecolor}, breakable, boxrule=0pt]}{\end{tcolorbox}}\fi

\hypertarget{test-of-standard-theorem-styles}{%
\section{Test of standard theorem
styles}\label{test-of-standard-theorem-styles}}

Ahlfors' Lemma gives the principal criterion for obtaining lower bounds
on the Kobayashi metric.

\begin{sta_ahlfors_s_lemma}\protect\hypertarget{ahlforsuxfffd--s-lemma}{}{}

Let \(ds^2 = h(z)|dz|^2\) be a Hermitian pseudo-metric on
\(\mathbf{D}_r\), \(h\in C^2(\mathbf{D}_r)\), with \(\omega\) the
associated \((1,1)\)-form. If
\(\mathop{\mathrm{Ric}}\nolimits\omega\geq\omega\) on \(\mathbf{D}_r\),
then \(\omega\leq\omega_r\) on all of \(\mathbf{D}_r\) (or equivalently,
\(ds^2\leq ds_r^2\)).

\end{sta_ahlfors_s_lemma}

\begin{lemma}[negatively curved families]

Let \(\{ds_1^2,\dots,ds_k^2\}\) be a negatively curved family of metrics
on \(\mathbf{D}_r\), with associated forms \(\omega^1\), \ldots,
\(\omega^k\). Then \(\omega^i \leq\omega_r\) for all \(i\).

\end{lemma}

Then our main theorem:

\begin{theorem}\protect\hypertarget{pigspan}{}{}

Let \(d_{\max}\) and \(d_{\min}\) be the maximum, resp.~minimum distance
between any two adjacent vertices of a quadrilateral \(Q\). Let
\(\sigma\) be the diagonal pigspan of a pig \(P\) with four legs. Then
\(P\) is capable of standing on the corners of \(Q\) iff
\begin{equation}\protect\hypertarget{eq-sdq}{}{\sigma\geq \sqrt{d_{\max}^2+d_{\min}^2}.}\label{eq-sdq}\end{equation}

\end{theorem}

\begin{corollary}

Admitting reflection and rotation, a three-legged pig \(P\) is capable
of standing on the corners of a triangle \(T\) iff Equation~\ref{eq-sdq}
holds.

\end{corollary}

\begin{remark}

As two-legged pigs generally fall over, the case of a polygon of order
\(2\) is uninteresting.

\end{remark}

\hypertarget{custom-theorem-styles}{%
\section{Custom theorem styles}\label{custom-theorem-styles}}

\begin{exercise}

Generalize Theorem~\protect\hyperlink{pigspan}{1.2} to three and four
dimensions.

\end{exercise}

\begin{note}

This is a test of the custom theorem style \texttt{note}. It is supposed
to have variant fonts and other differences.

\end{note}

\begin{bthm}

Test of the `linebreak' style of theorem heading.

\end{bthm}

This is a test of a citing theorem to cite a theorem from some other
source.

\begin{varthm}[Theorem 3.6 in Dummy (1900)]

No hyperlinking available here yet but that's not a bad idea for the
future.

\end{varthm}

\hypertarget{the-proof-environment}{%
\section{The proof environment}\label{the-proof-environment}}

\begin{proof}

Here is a test of the proof environment.

\end{proof}

\begin{proof}[Proof of Theorem \protect\hyperlink{pigspan}{1.2}]

And another test.

\end{proof}

\begin{proof}[Proof of \emph{necessity}]

And another.

\end{proof}

\begin{proof}[Proof of \emph{sufficiency}]

And another, ending with a display:

\[1+1=2\,. \qedhere\]

\end{proof}

\hypertarget{references}{%
\section*{References}\label{references}}
\addcontentsline{toc}{section}{References}

\hypertarget{refs}{}
\begin{CSLReferences}{1}{0}
\leavevmode\vadjust pre{\hypertarget{ref-thatone}{}}%
Dummy, D. 1900. {``Dummy Reference.''} \emph{Journal} 1 (1): 1--10.
\url{https://doi.org/10.1038/171737a0}.

\end{CSLReferences}



\end{document}
