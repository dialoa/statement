% Options for packages loaded elsewhere
\PassOptionsToPackage{unicode}{hyperref}
\PassOptionsToPackage{hyphens}{url}
%
\documentclass[
]{article}
\usepackage{amsmath,amssymb}
\usepackage{lmodern}
\usepackage{iftex}
\ifPDFTeX
  \usepackage[T1]{fontenc}
  \usepackage[utf8]{inputenc}
  \usepackage{textcomp} % provide euro and other symbols
\else % if luatex or xetex
  \usepackage{unicode-math}
  \defaultfontfeatures{Scale=MatchLowercase}
  \defaultfontfeatures[\rmfamily]{Ligatures=TeX,Scale=1}
\fi
% Use upquote if available, for straight quotes in verbatim environments
\IfFileExists{upquote.sty}{\usepackage{upquote}}{}
\IfFileExists{microtype.sty}{% use microtype if available
  \usepackage[]{microtype}
  \UseMicrotypeSet[protrusion]{basicmath} % disable protrusion for tt fonts
}{}
\makeatletter
\@ifundefined{KOMAClassName}{% if non-KOMA class
  \IfFileExists{parskip.sty}{%
    \usepackage{parskip}
  }{% else
    \setlength{\parindent}{0pt}
    \setlength{\parskip}{6pt plus 2pt minus 1pt}}
}{% if KOMA class
  \KOMAoptions{parskip=half}}
\makeatother
\usepackage{xcolor}
\IfFileExists{xurl.sty}{\usepackage{xurl}}{} % add URL line breaks if available
\IfFileExists{bookmark.sty}{\usepackage{bookmark}}{\usepackage{hyperref}}
\hypersetup{
  pdftitle={Label and info syntax tests},
  pdfauthor={Julien Dutant},
  hidelinks,
  pdfcreator={LaTeX via pandoc}}
\urlstyle{same} % disable monospaced font for URLs
\setlength{\emergencystretch}{3em} % prevent overfull lines
\providecommand{\tightlist}{%
  \setlength{\itemsep}{0pt}\setlength{\parskip}{0pt}}
\setcounter{secnumdepth}{-\maxdimen} % remove section numbering
\newlength{\cslhangindent}
\setlength{\cslhangindent}{1.5em}
\newlength{\csllabelwidth}
\setlength{\csllabelwidth}{3em}
\newlength{\cslentryspacingunit} % times entry-spacing
\setlength{\cslentryspacingunit}{\parskip}
\newenvironment{CSLReferences}[2] % #1 hanging-ident, #2 entry spacing
 {% don't indent paragraphs
  \setlength{\parindent}{0pt}
  % turn on hanging indent if param 1 is 1
  \ifodd #1
  \let\oldpar\par
  \def\par{\hangindent=\cslhangindent\oldpar}
  \fi
  % set entry spacing
  \setlength{\parskip}{#2\cslentryspacingunit}
 }%
 {}
\usepackage{calc}
\newcommand{\CSLBlock}[1]{#1\hfill\break}
\newcommand{\CSLLeftMargin}[1]{\parbox[t]{\csllabelwidth}{#1}}
\newcommand{\CSLRightInline}[1]{\parbox[t]{\linewidth - \csllabelwidth}{#1}\break}
\newcommand{\CSLIndent}[1]{\hspace{\cslhangindent}#1}
\usepackage{amsthm}
\theoremstyle{plain}
\newtheorem{theorem}{THEOreme}
\theoremstyle{plain}
\newtheorem*{klein�__s_lemma}{Klein's lemma}
\theoremstyle{plain}
\newtheorem*{klein�__s_lemma-1}{Klein's lemma}
\theoremstyle{plain}
\newtheorem*{klein�__s_lemma-2}{Klein's lemma}
\theoremstyle{plain}
\newtheorem*{klein�__s_lemma-3}{Klein's lemma}
\theoremstyle{plain}
\newtheorem*{klein�__s_lemma-4}{Klein's lemma}
\theoremstyle{plain}
\newtheorem*{principal_principle}{Principal Principle}
\theoremstyle{plain}
\newtheorem*{principal_principle-1}{Principal Principle}
\theoremstyle{plain}
\newtheorem*{principal_principle-2}{Principal Principle}
\ifLuaTeX
  \usepackage{selnolig}  % disable illegal ligatures
\fi

\title{Label and info syntax tests}
\author{Julien Dutant}
\date{}

\begin{document}
\maketitle

\begin{theorem}

This statement has no label or info.

\end{theorem}

\begin{theorem}[some info]

This statement has only info.

\end{theorem}

\begin{theorem}[some info]

This statement has only info, dot is allowed too.

\end{theorem}

\begin{theorem}[Dummy (1900; Otherdummy 1900)]

This statement has a couple of citations as info.

\end{theorem}

\begin{theorem}[Dummy (1900; Otherdummy 1900)]

This works with a point too.

\end{theorem}

\begin{theorem}[some info]

Info can be placed within Strong emphasis too.

\end{theorem}

\begin{theorem}[Dummy (1900)]

And a dot can be placed within the Strong emphasis.

\end{theorem}

\begin{theorem}[some info]

Or just outside.

\end{theorem}

\begin{klein�__s_lemma}

\protect\hypertarget{kleinuxfffd--s-lemma}{}{}This statement has a
custom label.

\end{klein�__s_lemma}

\begin{klein�__s_lemma-1}

\protect\hypertarget{KL}{}{}This statement has a custom label and
acronym.

\end{klein�__s_lemma-1}

The acronym can be used to crossrefer the statement
\protect\hyperlink{KL}{ KL }.

\begin{klein�__s_lemma-2}

\protect\hypertarget{KL}{}{}The filter doesn't care whether there is a
dot after or before the Strong emphasis label.

\end{klein�__s_lemma-2}

\begin{klein�__s_lemma-3}

\protect\hypertarget{KL}{}{}Or no dot at all.

\end{klein�__s_lemma-3}

\begin{klein�__s_lemma-4}

\protect\hypertarget{KL}{}{}Even if there's no space - ugly but we won't
chase it.

\end{klein�__s_lemma-4}

\begin{principal_principle}[Lewis]

\protect\hypertarget{PP}{}{}Info can be placed with the Strong emphasis
custom label.

\end{principal_principle}

\begin{principal_principle-1}[Dummy (1900; Otherdummy 1900)]

\protect\hypertarget{PP}{}{}Here the info is a Cite element within the
label.

\end{principal_principle-1}

\begin{principal_principle-2}[Dummy (1900; Otherdummy 1900)]

\protect\hypertarget{PP}{}{}Here the info is a Cite element without the
label.

\end{principal_principle-2}

\begin{theorem}

\textbf{(PP) (Lewis).} Acronym plus info without custom label should
fail. \texttt{**(PP)\ (Lewis).**} is treated is part of the theorem.

\end{theorem}

\hypertarget{references}{%
\section*{References}\label{references}}
\addcontentsline{toc}{section}{References}

\hypertarget{refs}{}
\begin{CSLReferences}{1}{0}
\leavevmode\vadjust pre{\hypertarget{ref-thatone}{}}%
Dummy, D. 1900. {``Dummy Reference.''} \emph{Journal} 1 (1): 1--10.

\leavevmode\vadjust pre{\hypertarget{ref-theother}{}}%
Otherdummy, A.N. 1900. {``Dummy Reference.''} \emph{Journal} 1 (1):
1--10.

\end{CSLReferences}

\end{document}
