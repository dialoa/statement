% Options for packages loaded elsewhere
\PassOptionsToPackage{unicode}{hyperref}
\PassOptionsToPackage{hyphens}{url}
%
\documentclass[
]{article}
\usepackage{amsmath,amssymb}
\usepackage{lmodern}
\usepackage{iftex}
\ifPDFTeX
  \usepackage[T1]{fontenc}
  \usepackage[utf8]{inputenc}
  \usepackage{textcomp} % provide euro and other symbols
\else % if luatex or xetex
  \usepackage{unicode-math}
  \defaultfontfeatures{Scale=MatchLowercase}
  \defaultfontfeatures[\rmfamily]{Ligatures=TeX,Scale=1}
\fi
% Use upquote if available, for straight quotes in verbatim environments
\IfFileExists{upquote.sty}{\usepackage{upquote}}{}
\IfFileExists{microtype.sty}{% use microtype if available
  \usepackage[]{microtype}
  \UseMicrotypeSet[protrusion]{basicmath} % disable protrusion for tt fonts
}{}
\makeatletter
\@ifundefined{KOMAClassName}{% if non-KOMA class
  \IfFileExists{parskip.sty}{%
    \usepackage{parskip}
  }{% else
    \setlength{\parindent}{0pt}
    \setlength{\parskip}{6pt plus 2pt minus 1pt}}
}{% if KOMA class
  \KOMAoptions{parskip=half}}
\makeatother
\usepackage{xcolor}
\IfFileExists{xurl.sty}{\usepackage{xurl}}{} % add URL line breaks if available
\IfFileExists{bookmark.sty}{\usepackage{bookmark}}{\usepackage{hyperref}}
\hypersetup{
  pdftitle={A sample document for the statement filter},
  pdflang={fr},
  hidelinks,
  pdfcreator={LaTeX via pandoc}}
\urlstyle{same} % disable monospaced font for URLs
\setlength{\emergencystretch}{3em} % prevent overfull lines
\providecommand{\tightlist}{%
  \setlength{\itemsep}{0pt}\setlength{\parskip}{0pt}}
\setcounter{secnumdepth}{5}
\ifLuaTeX
\usepackage[bidi=basic]{babel}
\else
\usepackage[bidi=default]{babel}
\fi
\babelprovide[main,import]{french}
% get rid of language-specific shorthands (see #6817):
\let\LanguageShortHands\languageshorthands
\def\languageshorthands#1{}
\usepackage{amsthm}
\newtheorem{theorem}{Théorème}[section]
\newtheorem{example}[theorem]{Example}
\newtheorem{lemma}[theorem]{Lemma}
\newtheorem{definition}[theorem]{Définition}
\ifLuaTeX
  \usepackage{selnolig}  % disable illegal ligatures
\fi
\usepackage[]{natbib}
\bibliographystyle{plainnat}

\title{A sample document for the statement filter}
\author{}
\date{}

\begin{document}
\maketitle

\hypertarget{mathematical-theorems}{%
\section{Mathematical theorems}\label{mathematical-theorems}}

\begin{example}

\begin{enumerate}
\def\labelenumi{(\alph{enumi})}
\tightlist
\item
  The set of all prime divisors of \(324\).
\item
  The set of all numbers divisible by 0.
\item
  The set of all continuous real-valued functions on the interval
  \([0,1]\).
\item
  The set of all ellipses with major axis \(5\) and eccentricity \(3\).
\item
  The set of all sets whose elements are natural numbers less than 20.
\end{enumerate}

\end{example}

\begin{example}[\citep{reference}]

another example

\end{example}

\(X \subseteq Y\) if and only if every element of \(X\) is an element of
\(Y\).

\hypertarget{more}{%
\section{More}\label{more}}

\textbf{The Axiom of Existence}. There exists a set which has no
elements.

If every element of \(X\) is an element of \(Y\) and every element of
\(Y\) an element of \(X\) then \(X=Y\).

\begin{lemma}

There exists only one set with no elements.

\end{lemma}

\begin{definition}

The (unique) set with no elements is called the empty set and denoted
\(\varnothing\).

\end{definition}

\begin{theorem}

A theorem.

\end{theorem}

\end{document}
