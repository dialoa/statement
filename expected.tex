% Options for packages loaded elsewhere
\PassOptionsToPackage{unicode}{hyperref}
\PassOptionsToPackage{hyphens}{url}
%
\documentclass[
]{article}
\usepackage{amsmath,amssymb}
\usepackage{lmodern}
\usepackage{iftex}
\ifPDFTeX
  \usepackage[T1]{fontenc}
  \usepackage[utf8]{inputenc}
  \usepackage{textcomp} % provide euro and other symbols
\else % if luatex or xetex
  \usepackage{unicode-math}
  \defaultfontfeatures{Scale=MatchLowercase}
  \defaultfontfeatures[\rmfamily]{Ligatures=TeX,Scale=1}
\fi
% Use upquote if available, for straight quotes in verbatim environments
\IfFileExists{upquote.sty}{\usepackage{upquote}}{}
\IfFileExists{microtype.sty}{% use microtype if available
  \usepackage[]{microtype}
  \UseMicrotypeSet[protrusion]{basicmath} % disable protrusion for tt fonts
}{}
\makeatletter
\@ifundefined{KOMAClassName}{% if non-KOMA class
  \IfFileExists{parskip.sty}{%
    \usepackage{parskip}
  }{% else
    \setlength{\parindent}{0pt}
    \setlength{\parskip}{6pt plus 2pt minus 1pt}}
}{% if KOMA class
  \KOMAoptions{parskip=half}}
\makeatother
\usepackage{xcolor}
\IfFileExists{xurl.sty}{\usepackage{xurl}}{} % add URL line breaks if available
\IfFileExists{bookmark.sty}{\usepackage{bookmark}}{\usepackage{hyperref}}
\hypersetup{
  pdftitle={A sample document for the statement filter},
  pdflang={fr},
  hidelinks,
  pdfcreator={LaTeX via pandoc}}
\urlstyle{same} % disable monospaced font for URLs
\usepackage{graphicx}
\makeatletter
\def\maxwidth{\ifdim\Gin@nat@width>\linewidth\linewidth\else\Gin@nat@width\fi}
\def\maxheight{\ifdim\Gin@nat@height>\textheight\textheight\else\Gin@nat@height\fi}
\makeatother
% Scale images if necessary, so that they will not overflow the page
% margins by default, and it is still possible to overwrite the defaults
% using explicit options in \includegraphics[width, height, ...]{}
\setkeys{Gin}{width=\maxwidth,height=\maxheight,keepaspectratio}
% Set default figure placement to htbp
\makeatletter
\def\fps@figure{htbp}
\makeatother
\setlength{\emergencystretch}{3em} % prevent overfull lines
\providecommand{\tightlist}{%
  \setlength{\itemsep}{0pt}\setlength{\parskip}{0pt}}
\setcounter{secnumdepth}{5}
\ifLuaTeX
\usepackage[bidi=basic]{babel}
\else
\usepackage[bidi=default]{babel}
\fi
\babelprovide[main,import]{french}
% get rid of language-specific shorthands (see #6817):
\let\LanguageShortHands\languageshorthands
\def\languageshorthands#1{}
\usepackage{amsthm}
\theoremstyle{definition}
\newtheorem*{example-unnumbered}{Exemple}
\theoremstyle{plain}
\newtheorem*{fact-unnumbered}{Fait}
\newtheoremstyle{empty}{1em}{1em}{\addtolength{\leftskip}{2em}\addtolength{\rightskip}{2em}}{0pt}{\scshape}{}{ }{}
\theoremstyle{empty}
\newtheorem*{statement}{}
\theoremstyle{definition}
\newtheorem*{the_axiom_of_existence}{The Axiom of Existence}
\theoremstyle{plain}
\newtheorem{theorem}{Théorème}[section]
\theoremstyle{definition}
\newtheorem{axiom}[theorem]{Axiome}
\theoremstyle{plain}
\newtheorem{lemma}[theorem]{Lemme}
\theoremstyle{definition}
\newtheorem{definition}[theorem]{Définition}
\providecommand{\proofname}{Démonstration}
\theoremstyle{plain}
\newtheorem{corollary}{Corollaire}[subsubsection]
\theoremstyle{definition}
\newtheorem*{named_principle}{Named principle}
\ifLuaTeX
  \usepackage{selnolig}  % disable illegal ligatures
\fi
\usepackage[]{natbib}
\bibliographystyle{plainnat}

\title{A sample document for the statement filter}
\author{}
\date{}

\begin{document}
\maketitle

\hypertarget{mathematical-theorems}{%
\section{Mathematical theorems}\label{mathematical-theorems}}

\begin{example-unnumbered}

\begin{enumerate}
\def\labelenumi{(\alph{enumi})}
\tightlist
\item
  The set of all prime divisors of \(324\).
\item
  The set of all numbers divisible by 0.
\item
  The set of all continuous real-valued functions on the interval
  \([0,1]\).
\item
  The set of all ellipses with major axis \(5\) and eccentricity \(3\).
\item
  The set of all sets whose elements are natural numbers less than 20.
\end{enumerate}

\end{example-unnumbered}

\begin{fact-unnumbered}[\citep{reference}]

another example

\end{fact-unnumbered}

Recursion test. he two following statements are within list items within
a Div element.

\edef\docparindent{\the\parindent}

\begin{enumerate}
\def\labelenumi{\arabic{enumi}.}
\item
  \begin{minipage}[t]{\textwidth}\parindent \docparindent

  \begin{statement}

  \addtolength{\rightskip}{2em}

  \(X \subseteq Y\) if and only if every element of \(X\) is an element
  of \(Y\).

  \end{statement}

  \end{minipage}\vskip .5em
\item
  \begin{minipage}[t]{\textwidth}\parindent \docparindent

  \begin{statement}

  \addtolength{\rightskip}{2em}

  \textbf{Named principle (NP)}. This is a named principle with an
  acronym.

  \end{statement}

  \end{minipage}\vskip .5em
\end{enumerate}

Crossreference test. See (\protect\hyperlink{NP}{}) or
(\protect\hyperlink{named-principle}{Named principle})! And citation
syntax (\protect\hyperlink{named-principle}{Named principle}).

\hypertarget{more}{%
\section{More}\label{more}}

\begin{figure}
\hypertarget{no-statement}{%
\centering
\includegraphics{image.jpg}
\caption{image}\label{no-statement}
}
\end{figure}

\begin{the_axiom_of_existence}

\protect\hypertarget{ax:existence}{}{}There exists a set which has no
elements.

\end{the_axiom_of_existence}

\hypertarget{subsec}{%
\subsubsection{subsec}\label{subsec}}

\begin{axiom}

\protect\hypertarget{sta:extensionality}{}{}If every element of \(X\) is
an element of \(Y\) and every element of \(Y\) an element of \(X\) then
\(X=Y\).

\end{axiom}

\begin{lemma}

There exists only one set with no elements.

\end{lemma}

\hypertarget{subsec-1}{%
\subsubsection{subsec}\label{subsec-1}}

\begin{definition}

The (unique) set with no elements is called the empty set and denoted
\(\varnothing\).

\end{definition}

\begin{proof}

Immediate from (axiom) and (axiom).

\end{proof}

\hypertarget{subsec-2}{%
\subsubsection{subsec}\label{subsec-2}}

\begin{corollary}

A corollary.

\end{corollary}

\begin{named_principle}

\protect\hypertarget{named-principle}{}{}This checks that two statements
with the same custom label get different environments.

\end{named_principle}

\end{document}
