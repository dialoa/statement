% Options for packages loaded elsewhere
\PassOptionsToPackage{unicode}{hyperref}
\PassOptionsToPackage{hyphens}{url}
%
\documentclass[
]{article}
\usepackage{amsmath,amssymb}
\usepackage{lmodern}
\usepackage{iftex}
\ifPDFTeX
  \usepackage[T1]{fontenc}
  \usepackage[utf8]{inputenc}
  \usepackage{textcomp} % provide euro and other symbols
\else % if luatex or xetex
  \usepackage{unicode-math}
  \defaultfontfeatures{Scale=MatchLowercase}
  \defaultfontfeatures[\rmfamily]{Ligatures=TeX,Scale=1}
\fi
% Use upquote if available, for straight quotes in verbatim environments
\IfFileExists{upquote.sty}{\usepackage{upquote}}{}
\IfFileExists{microtype.sty}{% use microtype if available
  \usepackage[]{microtype}
  \UseMicrotypeSet[protrusion]{basicmath} % disable protrusion for tt fonts
}{}
\usepackage{xcolor}
\IfFileExists{xurl.sty}{\usepackage{xurl}}{} % add URL line breaks if available
\IfFileExists{bookmark.sty}{\usepackage{bookmark}}{\usepackage{hyperref}}
\hypersetup{
  pdftitle={A sample document for the statement filter},
  hidelinks,
  pdfcreator={LaTeX via pandoc}}
\urlstyle{same} % disable monospaced font for URLs
\setlength{\emergencystretch}{3em} % prevent overfull lines
\providecommand{\tightlist}{%
  \setlength{\itemsep}{0pt}\setlength{\parskip}{0pt}}
\setcounter{secnumdepth}{-\maxdimen} % remove section numbering
\usepackage{amsthm}
\newtheoremstyle{empty}%
      {1em} % space above
      {1em} % space below
      {\addtolength{\leftskip}{2em}\addtolength{\rightskip}{2em}} % body font
      {0pt} % first line indentation (empty = no indent)
      {} % theorem head font
      {} % punctuation after theorem head
      {0pt} % space after theorem head
      {{}} % theorem head spec
    
\newtheoremstyle{emptynoindent}%
      {1em} % space above
      {1em} % space below
      {\addtolength{\leftskip}{2em}\addtolength{\rightskip}{2em}\parindent 0pt} % body font
      {0pt} % first line indentation (empty = no indent)
      {} % theorem head font
      {} % punctuation after theorem head
      {0pt} % space after theorem head
      {{}} % theorem head spec
    
\theoremstyle{empty}
\newtheorem{statement}{}
\theoremstyle{plain}
\newtheorem{corollary}{Corollary}
\theoremstyle{plain}
\newtheorem{principle}{Principle}
\theoremstyle{emptynoindent}
\newtheorem{argument}{}
\ifLuaTeX
  \usepackage{selnolig}  % disable illegal ligatures
\fi

\title{A sample document for the statement filter}
\author{}
\date{}

\begin{document}
\maketitle

A simple statement. {[}Here is some dummy text to show the normal line
length of a text paragraph in LaTeX.{]} Should be empty style:

\begin{statement}

This material is indented left and right. To see this we add a very long
line that will need to be broken at some point or other.

The second paragraph has a first line indent.

\end{statement}

A statement of the kind \texttt{corollary}, which is defined in the
defaults.

\begin{corollary}

This is a corollary. The kind is defined by default.

\end{corollary}

A statement of the kind \texttt{principle}, which has to be created on
the fly.

\begin{principle}[Quine’s principle]

Everything is something.

\end{principle}

A stament in the argument style, with an horizontal line in the
statement:

\begin{argument}

All is one.

One is less.

\nopagebreak[4]\raisebox{.25\baselineskip}{\rule{0.5\linewidth}{0.5pt}}\nopagebreak[4]

All is less.

\end{argument}

In LaTeX, statements that begin a list item normally create an empty
line. We avoid this by putting them into a minipage.

\edef\docparindent{\the\parindent}

\begin{itemize}
\item
  \begin{minipage}[t]{\textwidth}\parindent \docparindent

  \begin{statement}

  \addtolength{\rightskip}{2em}

  This starts a statement. To check the right indent we add a very long
  line that will need to be broken at some point or other. To check the
  right indent we add a very long line that will need to be broken at
  some point or other. To check the right indent we add a very long line
  that will need to be broken at some point or other.

  Statements second paragraphs are indented.

  \end{statement}

  \end{minipage}\vskip \baselineskip

  This is more text in the item.

  And even more text.
\item
  This item has normal text

  \begin{statement}

  Followed by a statement. To check the right indent we add a very long
  line that will need to be broken at some point or other. To check the
  right indent we add a very long line that will need to be broken at
  some point or other. To check the right indent we add a very long line
  that will need to be broken at some point or other.

  \end{statement}

  and more normal text.
\item
  \begin{minipage}[t]{\textwidth}\parindent \docparindent

  \begin{argument}

  \addtolength{\rightskip}{2em}

  This starts and argument

  \nopagebreak[4]\raisebox{.25\baselineskip}{\rule{0.5\linewidth}{0.5pt}}\nopagebreak[4]

  Argument paragraphs aren't indented.

  \end{argument}

  \end{minipage}\vskip \baselineskip

  This is more text in the item.
\end{itemize}

This list is a control to check that lists without statements are left
without changes:

\begin{enumerate}
\def\labelenumi{\arabic{enumi})}
\item
  Some random text in a Div
\item
  Another list entry
\item
  more text in a Div
\end{enumerate}

\end{document}
